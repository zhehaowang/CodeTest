\documentclass{article}

% the options fields are case sensitive
\usepackage[portuguese]{babel}
\usepackage[latin1]{inputenc}
\usepackage[T1]{fontenc}

% encoding options utf8 will allow the typesetting for chinese characters?
% \usepackage[utf8]{inputenc}
% ok, unfortunately it does not...XeLaTeX with XeCJK seems to be a more popular choice

\begin{document}
\par
This is a test for ``double quotation'' mark, and `single quotation' mark.
\par
See how default math fonts differ from normal font? 1 and $1$, 2 and $2$.
\par
See how the tildes differ: \~{} and $\sim$. The language rule in package babel is defined as French, notice that space before colon? It's not there in the source!
\par
Sometimes na\''{i}ve, na\''\i ve, na\``ive, na\"\i ve, Schr\"odinger, \textbackslash' adds a dot to its trailing character.
\par
Let's debug a long English paragraph, and see how different spacing between words works. See box alignment in action!
\section{Section vs part}
\par
\begin{verbatim}
After verbatim, having tons of spaces between    stuff gets printed. \\Latex commands gets printed \ldots...
\end{verbatim}
\par
Whose woods these are I think I know.   
His house is in the village though; 
He will not see me stopping here   
To watch his woods fill up with snow.   
My little horse must think it queer   
To stop without a farmhouse near   
Between the woods and frozen lake   
The darkest evening of the year.   
\part{Part vs paragraph}
He gives his harness bells a shake   
To ask if there is some mistake.   
The only other sound's the sweep   
Of easy wind and downy flake.   
The woods are lovely, dark and deep,   
But I have promises to keep,   
And miles to go before I sleep,   
And miles to go before I sleep.
%如果我在源文件中打入中文,而不指定font/input encoding,会出问题么?是会的。

\begin{tabular}{|r|l|c|}
\hline
this is a line & random stuff & and this is central alignment\\
this is another line & see how it aligns & the | in the table parameter helps \\
\hline
\end{tabular}

\par
Folllowing is a trick for float aligning in latex tabulars.

\par
\begin{tabular}{c|r @{.} l}
Pi expression       &
\multicolumn{2}{c}{Value} \\
\hline
$\pi$               & 3&1416  \\
$\pi^{\pi}$         & 36&46   \\
$(\pi^{\pi})^{\pi}$ & 80662&7 \\
\end{tabular}

\end{document}